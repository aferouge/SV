\section{Model verification}
\label{sec:check}

In section~\ref{sec:trans} the translations of the requirements are also
translated into $\mu$-calculus. From that it is trivial to transform that into
a mcf (mu-calculus formula) file which can be checked with the mCRL2 tools. A
mcf file can be checked against a \texttt{lps} file with the following command:

\begin{verbatim}
lps2pbes bridge.lps -f check01.mcf | pbes2bool
\end{verbatim}

The output of this command is either true or false. Each check is constructed such
that the output of the command should be \texttt{true} for each mcf file.

After writing all the requirements into the mcf format we have found that all
the model satisfies all $\mu$-calculus formulas, as stated in
table~\ref{tab:checks}. The checks correspond with the translated requirements
from section~\ref{sec:trans}, and can be found in appendix~\ref{sec:mcf}. The
\texttt{false-test} is an extra check to make sure at least once \texttt{false}
is returned: \texttt{[true*]false}. The deadlock free test is a test to check
that there is no deadlock in the model: \texttt{[true*]<true>true}.

\begin{table}[htb]%
\centering
\begin{tabular}{|c|l|}
	\hline
	\textbf{Requirement} & \textbf{Verification}\\
	\hline
	1 & Fully satisfied\\
	2 & Fully satisfied\\
	3 & Fully satisfied\\
	4 & Fully satisfied\\
	5 & Fully satisfied\\
	6 & Fully satisfied\\
	7 & Fully satisfied\\
	8 & Fully satisfied\\
	9 & Fully satisfied\\
	10 & Fully satisfied\\
	11 & Fully satisfied\\
	\hline
	false-test & Not satisfied\\
	deadlock free & Fully satisfied\\
	\hline
\end{tabular}
\caption{Verification of the safety requirements in the mCRL2 model of the bridge control system.}
\label{tab:checks}
\end{table}

