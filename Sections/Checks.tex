\section{Model verification}
\label{sec:check}

In section~\ref{sec:trans} the translations of the requirements are also
translated into $\mu$-calculus. From that it is trivial to transform that into
a mcf (mu-calculus formula) file which can be checked with the mCRL2 tools. For example the first
requirement of~\ref{sec:trans} is translated to:

\begin{verbatim}
[true* . commPreSign(off) . !commPreSign(on)* . commStopSign(on)]false
\end{verbatim}

After writing all the requirements into the mcf format we have found that all
the model satisfies all $\mu$-calculus formulas, as states in table \ref{tab:checks}.
%
\begin{table}[htb]%
\centering
\begin{tabular}{|c|l|}
	\hline
	\textbf{Requirement} & \textbf{Verification}\\
	\hline
	1 & Fully satisfied\\
	2 & Fully satisfied\\
	3 & Fully satisfied\\
	4 & Fully satisfied\\
	5 & Fully satisfied\\
	6 & Fully satisfied\\
	7 & Fully satisfied\\
	8 & Fully satisfied\\
	9 & Fully satisfied\\
	10 & Fully satisfied\\
	\hline
	false-test & Not satisfied\\
	deadlock free & Fully satisfied\\
	\hline
\end{tabular}
\caption{Verification of the safety requirements in the mCRL2 model of the bridge control system.}
\label{tab:checks}
\end{table}

\newpage