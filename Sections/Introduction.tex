\subsection{Introduction}

As has been pointed out by the incident at the Ketelbrug between Emmeloord and Lelystad, bridges require a very precise control system.  Many examples exist where bridges are not opened or closed properly, where bridge components break down or the bridge is being stuck in a particular state. The goal of this project is to design a satefy layer that can be added to the main control interface, in order to capture all possible situations and failures.
Throughout the report, we make use of the following assumptions:
%
\begin{itemize}
	\item If three or four sensors detect an open/closed bridge, the bridge can be considered open/closed
	\item If the four sensors	 have a 50/50 detection on the bridge deck, the bridge should remain in its current position (open/closed)
	\item A barrier can only be considered to be down when the majority of the sensors detect a lowered barrier
\end{itemize}
%
In this section, the desired behaviour of the bridge is being determined and defined. Section \ref{sec:glob} contains the global requirements that will be met in our system. Specific interactions that can be performed by the bridge are discussed in Section \ref{sec:act}, whereas Section \ref{sec:trans} translates these interactions into the requirements stated earlier.
