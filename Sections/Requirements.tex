\section{Global Requirements}
\label{sec:glob}

Having used the project guide as a source for the desired behaviour, the global safety requirements are defined. First, safety requirements considering the opening sequence of events are described. Secondly, the same is done for the closing procedure. Finally, some obvious functional requirements are defined to ensure the bridge is capable of basic up and down movements.

\subsubsection*{Opening the bridge}
\begin{enumerate}
	\item Stop signs cannot be switched on unless all pre signs are switched on
	\item Barriers cannot be lowered unless all stop signs are switched on
	\item The first barrier to be encountered by the cars is lowered earlier than the second in order to enable cars to leave the bridge
	\item The bridge can only be unlocked when all barriers are down
	\item The deck can only be lifted when both locks are disengaged
	\newcounter{enumTemp}
	\setcounter{enumTemp}{\theenumi}
\end{enumerate}

\subsubsection*{Closing the bridge}
\begin{enumerate}
	\setcounter{enumi}{\theenumTemp}
	\item Bridge locks can only be engaged when the deck is down
	\item Barriers can only be up when the bridge is locked
	\item Stop signs can only be shut off only when all barriers are up
	\setcounter{enumTemp}{\theenumi}
\end{enumerate}

\subsubsection*{Functional requirements}
\begin{enumerate}
	\setcounter{enumi}{\theenumTemp}
	\item The bridge should be able to be opened when a ship wants to pass
	\item The bridge should be able to close in order to let cars pass
	\setcounter{enumTemp}{\theenumi}
\end{enumerate}


\subsubsection*{Failure}
\begin{enumerate}
	\setcounter{enumi}{\theenumTemp}
%	\item The stop signs can only be lit if at least one pre sign is lit at
%	      each side of the bridge
%	\item The barriers can only be lowered if both stop signs are lit at each
%	      side of the bridge
	\item If the motor is in the `broken' status, the deck may not be moved up or down
%	\setcounter{enumTemp}{\theenumi}
\end{enumerate}
%

